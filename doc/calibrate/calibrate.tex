\section{Calibrate}
\todo{a introduction}

\subsection{The axies}
\todo{Notes on how to Calibrate the x,y,z}

\subsubsection{The Z axies with autolevel sensor}
This methode I have been using and it works quite well for me. 
I use it when, there has been a nozzle change or a new fiberplade install. \\
\inforeader{To do this the printer needs to have a display with encoder on.}
G-code commands needed to be sendt from computer or octoprint to the printer:
\begin{itemize}
    \item G28
    \item M851 Z0
    \item M500
    \item M501
    \item M851
    \item \todo{delay}
    \item G28 Z
    \item G1 F60 Z0
    \item \todo{explaine gcode}
    \item M211 S0; Disable software endstops
    \item M851 Z{new Z value on the display}
    \item M211 S1; Enable software endstops
    \item M500; Save the value to eeprom
    \item M501; Read the value from eeprom
\end{itemize}



\subsection{The extruder}
\todo{Notes on calibrate the extruder}